%
There is ample evidence about the effects and importance of the nonlinear response of sediments during strong earthquake ground motion. Depending on the modeling scales and the level of knowledge about material properties, these effects can be estimated by means of nonlinear or equivalent linear methods. While the former is well-founded in theoretical grounds and the known (experimental) behavior of geomaterials, the latter is based on simplified assumptions. The equivalent linear method is, nonetheless, more broadly used in engineering practice due to its simplicity and effectiveness. In principle, this method is limited to problems under shear-deformation and the one-dimensional (1D), vertical propagation of seismic waves through horizontal layers. Realistic, nonlinear soil-modeling approaches, on the other hand, while more appropriate for general applications, are restricted to highly specialized studies. This is mostly due to the required level of knowledge about material properties and the entailed computational overhead. As a result, three-dimensional (3D) nonlinear earthquake ground motion simulations at regional scales are seldom done. In this study we explore the applicability of equivalent linear-method ideas in 3D earthquake ground motion simulations at a regional scale. We implement a version of the linear equivalent method in 3D using a finite element software for wave propagation simulations, and start by testing our implementation under 1D (shear) wave propagation conditions. We then follow a careful, progressive approach to study the applicability of well-known shear modulus degradation and damping ratio curves to update material properties during the iterative simulation process, as we increase the geometrical complexity of the models from 1D to 3D conditions. We compare the results obtained with nonlinear solutions, analyze the qualitative effectiveness of the approach to approximate the known effects and characteristics of nonlinear behavior in 3D basins during earthquake ground shaking, and draw conclusions from these comparisons to inform possible paths forward.