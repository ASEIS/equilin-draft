
\section{Draw backs of equivalent linear method}

All site response analyses are burdened with significant uncertainties in the parameters, models, and methodologies. 
Although equivalent linear method in practice is the most used method, however, like many other methods it has advantages and disadvantages. In this section we will discuss these factors. Equivalent linear method, in nature, is a linear process. Therefore, it does not have nonlinear displacement. Consequently the method is not good for studying permanent displacement due to nonlinear soil behavior. Also as mentioned before, for large strain levels that can affect the damping ratio and shear modulus with progression of cycles in load application, the equivalent linear method is not a good approach. Equivalent linear method is good for estimating long-period (low frequency) motions, however, it is not good for high frequency motions. The reason for this could be using the maximum strain or a factor of maximum strain to adjust the values. Maximum strains are results of long period motions so the results are accurate for them \citep{Kaklamanos2013critical}. 
As I mentioned above, the equivalent linear method has an overdamping issue for high frequency motions. \citep{Sugito1994frequency,Joyner1998equivalent,Assimaki2002equivalent,Park2008rate} worked on the high frequency over damping issue. However, the recommended details are to much complicated that is not in engineering practice yet and people prefer to do nonlinear site response analysis rather than that. \citet{Assimaki2008quantifying} suggests when the rock-outcrop (input) peak ground acceleration (PGA) exceeds 0.2g for soft (NEHRP class E) sites, fully nonlinear analysis is necessary. \citet{Kim2013site} found that the accuracies of equivalent linear model and nonlinear models were generally similar, but the prediction deviate when maximum shear strains exceed 0.3\%. \citet{Yee2013elastic} found that equivalent linear and nonlinear models offer similar predictions for maximum shear strains up to 0.2\%, but that prediction deviate at greater strains. When shear strains in the soil exceed some critical level, the equivalent linear approximation becomes inadequate, and fully nonlinear site response analysis are needed to accurately predict surface ground motion. See following references: \citep{Assimaki2008quantifying,Kwok2008,Kaklamanos2013critical,Kim2013site,Yee2013elastic,Matasovic2010,Hashash2016deepsoil}\\

While, however, nonlinear models are necessary when large strains and permanent displacements are expected, their prediction accuracy depends on the constitutive material law that governs soil behavior. Elaborate constitutive laws require numerous parameters that need to be determined through laboratory and field techniques and are therefore associated with considerable cost and effort in design practice because they require both detailed site characterization and significant engineering time for analysis \citep{Assimaki2008quantifying}.

Despite the effectiveness of the approach for the analysis of relatively stiff sites subjected to immediate levels of strain $(<10^{-3})$, however, the equivalent linear method has been shown to over estimate the peak ground acceleration for large events and artificially suppress the high frequency components when applied for the analysis of deep sites \citep{Assimaki2008quantifying}.

An alternative methodology that accounts for the frequency dependence of strain amplitudes and associated dynamic soil properties has been proposed by \citet{Assimaki2002equivalent}, and it has been shown to yield more satisfactory results for deep sedimentary deposits; the applicability of the alternative formulation, however, is still limited to the medium strain levels (See \citet{Assimaki2008quantifying}).

\citet{Cramer2004memphis} generated a suite of seismic hazard maps for Memphis, Shelby county, Tennessee that account for the site response of sediments in the Mississippi embayment (ME). They showed that SHAKE91 over damp the high frequency components of motion. Nonlinear methods are necessary to capture phenomena such as irreversible deformations and pore pressure coupling \citep{Assimaki2008quantifying}.


%Engineers often perform ELA rather that NLA because of their simplicity of use and low computational requirements \citep{Matasovic2012practices}
%\citet{Yee2013elastic,Tsai2009learning,Elgamal2001dynamic} suggested that the soil damping derived from laboratory testing may be too low to accurately reflect in situ damping behavior. The reason for these damping misfits are unknown but are likely related to modeling a three dimensional system in one dimensional, which necessarily omits complexities that may introduce additional damping to the site response.


