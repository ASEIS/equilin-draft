
\section{Introduction}

Near surface low velocity layer has a significant effect on seismic waves frequency content and amplitude. Amplification, deamplificaiton and shift in resonant frequency are important factors in designing structures and buildings. Devastating damage after Micheocan, Mexico (1985), Kalmat, Greece (1989), Loma Perieta, California, USA (1989), Roodbar-Manjil, Iran (1990) and other earthquakes highlighted the importance of comprehensive site response analyses in engineering practices. Fully nonlinear and equivalent linear analyses are used to consider the nonlinear behavior of soil during earthquake. Among those the latter lend itself in engineering practices. Fully nonlinear site response analysis is important to estimate soil behavior during earthquake and are necessary to capture phenomena such as irreversible deformations and pore pressure coupling, however, its usage in standard engineering practice is relatively limited \citep{Assimaki2008quantifying}. The main reason is lack of soil data.  Nonlinear soil analysis needs parameter which is not available for all site even for well stablished KiK-net sites \citep{Kaklamanos2013critical}. Elaborate constitutive laws require numerous parameters that need to be determined through laboratory and  field techniques and are therefore associated with considerable cost and effort in design practice because they require both detailed site characterization and significant engineering time for analysis \citep{Assimaki2008quantifying}.

The equivalent linear method is perhaps the most widely employed approach for strong motion site response predictions in engineering practice\citep{Assimaki2008quantifying}. The idea of replacing a nonlinear system by an "equivalent" linear system whose behavior will be an approximation to that of the nonlinear system was first introduced by \citet{jacobsen1930steady}. He compared the various criteria of equivalence with exact solutions for specific problems,  and with experimental results, He concluded that the most useful criterion is that of the equivalence of energy dissipated per cycle. In other study, \citet{Kryloff1943} replaced the general nonlinear equations' damping and spring constant with modified values so that the solution of the linear equation is an approximate solution of the nonlinear equation. \citet{Hudson1965}  defined an equivalent viscous damping coefficient by equating the energy loss per cycle in a hysteretic system to energy loss per cycle in the corresponding small amplitude linear system. The new damping coefficient resulted in accurate calculations of the maximum amplitude of resonant vibrations in hysterestic systems that are strongly nonlinear.  Based on these successful applications of equivalent linear analyses, \citet{Idriss1968} proposed the implementation of the equivalent linear method in site response analysis. The procedure involves the determination of an equivalent linear modulus, $G_{eq}$, and an equivalent damping ratio $\lambda_{eq}$ for use in a linear elastic solution. They proposed to use other numerical methods (e.g., Finite-Element method) in case of  non-horizontal boundaries, and non horizontal excitations. \citet{Idriss1970} summarized the available data of shear modulus and damping for different soil and site and also different tests  and  developed  shear moduli and damping ratios for soils with different parameters. Integrating all these theory and laboratory results,  \citet{Schnabel1972a} developed a software known as SHAKE to conduct a 1D site response analysis for horizontal layers. Equivalent linear methods because of simplicity, need for less input parameters, and computationally efficiency have been used in many different engineering practice and site response analysis. Two decades later, \citet{Idriss1992}, developed the modified version of SHAKE and called it SHAKE91. The modified version of SHAKE was suitable for personal computers. The program computes the response of  semi-infinite horizontally layered soils deposit overlying a uniform half-space subjected to vertically propagating shear waves. The analysis is done in the frequency domain, and, therefore, for any set of properties it is a linear analysis. SHAKE91 still is in use in engineering practice and research. Since then, equivalent linear method, specifically shake is used in numerous site response analysis. SHAKE is the most widely used analysis package for 1D site-specific response calculations \citet{Assimaki2008quantifying}. \\
Many other packages were developed to conduct site response analysis using equivalent linear method, however, the general idea is the same. (give some references).


Equivalent linear method are mostly used through 1D site response analysis. However, the idea is used in 2D and 3D simulation to approximate the nonlinear soil behavior. \citet{Lysmer1974} and \citet{Lysmer1975} were among the initial attempts to include the equivalent linear method in 3D simulation. They developed a computer program (LUSH and FLUSH) to approximate 3D analysis of soil-structure interaction problems mostly for nuclear power plant structures. A control motion,  specified at some point in the free-field surface profile, can be deconvolved to determine the corresponding motions at some depth, such as a soil-rock interface. The motion computed at this depth is used as input to a finite element model of the soil structure system. In each iteration the analysis is linear but the soil properties are adjusted from iteration to iteration until the computed strains are compatible with the soil properties used in the analysis. It mostly needs 3 to 5 iterations to converge. They proposed a method to define the effective strain based on current strain level and input acceleration level. In another study, \citet{Lysmer1981} developed a system for 3D soil-structure interaction to analyze the horizontal viscoelastic layers on horizontal viscoelastic half space through accepting a combination of incoming body wave and horizontal surface wave for standard 3D finite element structures. They approximated the nonlinear effects by an equivalent linear method.  

Full 3D nonlinear soil simulation with addressing the 3D aspects present in a region where soil nonlinearities are combined with source path and basin effects is studied from different perspective. \citet{Xu2003} reported the development and application of a parallel numerical methodology for simulating large-scale earthquake-induced ground motion in highly heterogeneous basin whose soil constituents can deform nonlinearly . They used an idealized basin with an arbitrary seismic excitation. The basin material is modeled as Drucker-Prager elastoplastic materials. Their simulations show that elastoplastic soil behavior results in overall reduction of the ground acceleration through the basin. The significant reduction in peak resonance was reported, however, there was small changes in dominant frequency. \cite{Dupros2010} presented finite-element numerical simulation of seismic wave propagation in nonlinear inelastic geological media. They used Mohr-Coulomb model for simulating the nonlinear behavior of soil. They conducted the simulation within French Riviera up to 0.5 Hz. They observed increment in peak ground displacement and reduction in peak ground velocity and acceleration due to nonlinear soil behavior. \citet{Taborda2010} implemented vonMises and Drucker-Prager models in 3D ground motion simulation FEM code called Hercules.  two cases for the elastoplastic ground response of the valley for moderate and severe nonlinearities.  (Also include Taborda Bielak Restrepo 2012).

As mentioned earlier nonlinear site response analysis needs elaborate set of soil parameters which is not available for all places. The analyses become more computationally challenging in 3D simulation. In this paper we propose using equivalent linear method in 3D ground motion simulation from source to site without further pre- or post-processing. We compare the fully nonlinear and equivalent linear simulation results and discuss the appropriate range of parameters that needs to be consider. In the following section we discuss the equivalent linear solution steps and implementation in 3D ground motion simulation. Later on we discuss the fully nonlinear solutions. We test the idea on different site, source, and magnitudes. We also use the simulation for realistic simulation. Our results show that the equivalent linear method have good agreement with fully nonlinear solution in terms of PGV, PGA, and their frequency contents. It has also a good agreement with response spectra. Regarding the significantly low computational cost, and lower number of input values, equivalent linear method in 3D simulation can be a good alternative for fully nonlinear methods in large scale ground motion simulation for considering nonlinear soil behavior. 






%The most common manifestations of inelastic soil behavior involve the reduction in shear wave velocity and the increase in soil damping with increasing load (Hardin and Drnevich, 1972). Simulations of nonlinear earthquake response be-gan in the late 1960s. These early studies were con- ducted for horizontally layered soils and vertically incident waves, by either an equivalent linear or a direct nonlinear method. These are the main meth- ods that are still used today in engineering practice. In the equivalent linear method, the soil response is evaluated in an iterative manner. First, trial values for average strain are chosen, then soil properties are de- termined in accordance with the trial values of strain, and finally the response of the model is calculated. If the calculated strains differ significantly from the trial values, the cycle is repeated (Idriss and Seed, 1968; Schnabel et al., 1972).While one-dimensional (1D) simulations can yield reasonable estimates of nonlinear effects under ver- tically incident seismic excitation, they cannot rep- resent the effects of surface waves and basin effects. However, observations from many recent strong motion events have demonstrated that nonlinear soil behavior strongly affects the seismic motion of near-surface deposits, resulting in shear wave velocity reduction, irreversible settlements, and in some cases pore-pres- sure build-up leading to liquefaction.The effect of the nonlinear constitutive law on acceleration is much more difficult to interpret; the global trend is to add an important high frequency content (in agreement with [43]) and to increase peak ground acceleration after the shear wave portion: see I. Beresnev, K. Wen, Nonlinear soil response ? a reality?, Bull Seism. Soc. Am. 86 (6) (1996) 1964?1978.

